\documentclass[12pt]{article}

\usepackage[utf8]{inputenc}
\usepackage{ngerman}
\usepackage{graphicx}


\title{Exposé \\[1ex] \large Thema: Wie verhindern Entwickler bildergenerierender KIs, dass diese nicht zur Erzeugung ethisch fragwürdiger Inhalte missbraucht werden.}
\date{02.04.2023}
\author{Ronja Drechsler, \and Dominic Fitter, \and Michel Hecker, \and Khaldon Kassem, \and Phillip Eckstein}

\begin{document}
\maketitle

\section{Thema}
Bildgenerierende künstliche Intelligenz (IGAI) bezieht sich auf Algorithmen und Technologien, die es Computern ermöglichen, Bilder und Grafiken auf der Grundlage von Daten und bestimmten Vorgaben zu erstellen. Dies geschieht durch den Einsatz von sogenannten \emph{neuronalen Netzen}, die in der Lage sind, Muster in Daten zu erkennen und diese Muster dann in Bilder zu übersetzen.

Diese Technologie wird in verschiedenen Anwendungen eingesetzt, beispielsweise bei der Erstellung von Kunstwerken, der Erstellung von visuellen Effekten in Filmen, der Erstellung von realistischen 3D-Modellen und sogar bei der Erstellung von medizinischen Bildern für diagnostische Zwecke.

Ein bekanntes Beispiel für bildgenerierende AI ist das sogenannte \emph{DeepDream}-Verfahren von Google, bei dem ein neuronales Netzwerk genutzt wird, um Bilder zu erzeugen, die surreal und traumhaft wirken. Ein weiteres Beispiel ist das \emph{GAN}-Verfahren (Generative Adversarial Networks), bei dem zwei neuronale Netze zusammenarbeiten, um Bilder zu erzeugen, die kaum von echten Bildern zu unterscheiden sind.

IGAIs können jedoch nicht nur positive Inhalte generieren. So werden sie gezielt zur Genereirung von gewaltverherlichenden oder pornografischen Inhalten missbraucht oder Biases führen dazu, dass rassistische Bilder genereiert werden.
\section{Forschungsfrage}
In dieser Arbeit soll die Frage geklärt werden, wie Entwickler der IGAIs verhindern, dass diese nicht zur Erzeugung ethisch fragwürdiger Inhalte missbraucht werden.
\section{Theoretische Grundlagen}
Um die Forschungsfrage zu beantworten soll geklärt werden welche IGAIs es gibt und wie diese bereits mit der Thematik unethischer Inhalte befassen, wobei sich in erster Linie auf die von IGAI-Anbietern bereitgestellten Informationen konzentriert werden soll: 
\begin{itemize}
    \item www.labs.openai.com/policies/content-policy
    \item www.openai.com/blog/reducing-bias-and-improving-safety-in-dall-e-2
    \item www.docs.midjourney.com/docs/community-guidelines

\end{itemize}
Zudem sollen Definitionen für ethisch fragwürdige Inhalte gegeben und diskutiert werden, inwiefern es notwendig ist, diese zu vermeiden.
\begin{itemize}
    \item www.researchgate.net/publication/351486496
    \item www.bpb.de/kurz-knapp/lexika/das-junge-politik-lexikon/320245/ethik/
    \item www.researchgate.net/publication/340115931
\end{itemize}
\section{Stand der Forschung}
Bildgenerierende KI ist für die Öffentlichkeit erst in den letzten Jahren aufgetaucht. Entsprechend dieses Umstandes, kann es im Verlauf der Ausarbeitung nötig werden, die Forschungsfrage vom praktischen Ansatz, den aktuellen und geplanten Entwicklungen, zu einem theoretischeren Ansatz, der theoreitsch möglichen Lösungen zur Begrenzung der erzeugbaren Inhalte, auszuweiten.
\section{Forschungsdesign}
Nach einer Definition von IGAIs und einer Beschreibung derer Funktionsweise wird definiert, was unethische bzw. ethisch fragwürdige Inhalte sind und deren Bedeutung für und Auswirkung auf die menschliche Kultur beleuchtet. Im Folgenden wird betrachtet, wie die Generierung unethischer Bilder bisher verhindert wird bzw. von den Anbietern verhindert werden soll. Basierend auf dem gewonnenen Wissen wird untersucht, wie mit IGAIs absichtlich unethisches Bildmaterial generiert werden kann.
\section{Zeitplan}
\begin{center}
    \begin{tabular}{ |l|l|l| } 
     \hline
     Aufgabe & Termin & Geplanter Zeitaufwand \\ 
     \hline
     Abgabe Exposé & 03.04.2023 & 6 Stunden \\ 
     \hline
     Abgabe Forschungsstand & 17.04.2023 & 20 Stunden \\ 
     \hline
     Abgabe Textentwurf & 01.05.2023 & 25 Stunden \\ 
     \hline
     Abgabe erster Forschungsbericht & 21.05.2023 & 30 Stunden \\ 
     \hline
     Abgabe Peer-Grading-Berichte & 29.05.2023 & 6 Stunden \\ 
     \hline
     Abgabe finaler Forschungsbericht & 26.06.2023 & 20 Stunden \\ 
     \hline
    \end{tabular}
    \end{center}
\end{document}