\documentclass[12pt]{article}

\usepackage[utf8]{inputenc}
\usepackage{ngerman}
\usepackage{graphicx}

\title{Expose zu}
\date{30.03.2023}
\author{Ronja Drechsler,Dominic Fitter,Michel Hecker,Khaldon Kassem,Phillip Eckstein}

\begin{document}
\maketitle

\section{Thema}
Bildgenerierende künstliche Intelligenz (AI) bezieht sich auf Algorithmen und Technologien, die es Computern ermöglichen, Bilder und Grafiken auf der Grundlage von Daten und bestimmten Vorgaben zu erstellen. Dies geschieht durch den Einsatz von sogenannten \emph{neuronalen Netzen}, die in der Lage sind, Muster in Daten zu erkennen und diese Muster dann in Bilder zu übersetzen.

Diese Technologie wird in verschiedenen Anwendungen eingesetzt, wie beispielsweise bei der Erstellung von Kunstwerken, der Erstellung von visuellen Effekten in Filmen, der Erstellung von realistischen 3D-Modellen und sogar bei der Erstellung von medizinischen Bildern für diagnostische Zwecke.

Ein bekanntes Beispiel für bildgenerierende AI ist das sogenannte \emph{DeepDream}-Verfahren von Google, bei dem ein neuronales Netzwerk genutzt wird, um Bilder zu erzeugen, die surreal und traumhaft wirken. Ein weiteres Beispiel ist das \emph{GAN}-Verfahren (Generative Adversarial Networks), bei dem zwei neuronale Netze zusammenarbeiten, um Bilder zu erzeugen, die kaum von echten Bildern zu unterscheiden sind.

Ein Nagtivbeispiel von PGAIs ist der Missbrauch dieser zur Generierung von gewaltverherlichenden oder pornografischen Inhalten. Außerdem können Biases dazu führen, dass rassistische Bilder generiert werden. 
\section{Forschungsfrage}
In dieser Arbeit soll die Frage geklärt werden wie Entwickler der PGAIs verhindern, dass diese nicht zur Erzeugung ethisch fragwürdiger Inhalte missbraucht werden. Dafür soll geklärt werden was genau ethisch fragwürdige Inhalte sind und warum diese schädlich sind.
\section{Theoretische Grundlagen}
Erste Recherchen zum Thema ergaben, dass sich im wissenschaftlichen Raum noch nicht weitgehend mit diesem Thema befassen wurde. Für weitere Recherche zum Thema dienen als erste Anlaufstellen die von den IGAI-Anbietern bereitgestellten Informationen.
Zudem soll eine Definition für ethisch fragwürdige Inhalte gegeben werden und ob es notwendig ist diese zu vermeiden.
\section{Stand der Forschung}
Bildgenerierende KI ist für die Öffentlichkeit erst in den letzten Jahren aufgetaucht. Dementsprechend gibt es kaum Forschung bezüglich der konkreten Forschungsfrage.
\section{Forschungsdesign}
Zuerst erfolgt eine Recherche zu IGAIs selbst, um definieren zu können was diese konkret sind und wie sie funktionieren. Parallel wird definiert was unethisch/ethisch fragwürdige Inhalte sind und deren Beudeutung für und Auswirkung auf die menschliche Kultur. Es werden Recherchen betrieben, um zu ermitteln wie die Generierung unethische Bilder verhindert wird. Basierend auf dem gewonnenen Wissen, werden Experimente an verschiedenen IGAIs durchgeführt, mit der Absicht unethisches Bildmaterial zu generieren.


\end{document}