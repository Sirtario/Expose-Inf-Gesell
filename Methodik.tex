\chapter{Methodik}

Die genutzten Quellen verwenden verschiedene Methodiken, um ihre Ergebnisse zu erreichen. Es wurden Experimente angefertigt wie in \cite{Zheng}. Hier wurde zum Beispiel ein Bild-Filter gegen pornografische Inhalte erstellt. Minikken \cite{Minkkinen} zeigt eine Nutzweranalyse. Des Weiteren wurden verschiedene Literaturübersichten angefertigt wie in \cite{Srinivasan} oder \cite{Jobin}. Auch in der vorliegenden Arbeit soll eine Übersicht anhand von Literatur erstellt werden. Im Unterschied zu den genannten Literaturübersichten liegt der Fokus jedoch darauf, die Mechanismen zu erfassen, die eingesetzt werden, um die unethische Verwendung von \GLSabrev{igai} zu verhindern. Als Grundlage hierfür werden zunächst \GLSabrev{igai} und Moral in Bezug auf diese \GLSabrev{ai}s definiert und zur Erläuterung der Relevanz von Moral beim Umgang mit \GLSabrev{ai} diskutiert. Danach werden die Mechanismen aus der betrachteten Literatur erfasst und es wird untersucht, an wann im Lebenszyklus einer \GLSabrev{igai} sie eingesetzt werden können oder sich die erarbeiteten Ansätze übertragen lassen.