\section{Methodik}
Die vorliegende Arbeit hat zum Ziel, eine umfassende Betrachtung der Begriffe 
bildgenerierende KI (IGAI) und Moral in 
Bezug auf diese KIs zu liefern und die Relevanz der Moral im Umgang mit KI
zu erläutern. Hierbei sollen die genannten Begriffe zunächst definiert und im Anschluss diskutiert werden. 
Des Weiteren wird in dieser Arbeit untersucht, an welchen konkreten 
Punkten im Lebenszyklus einer IGAI, von der Entwicklung bis hin zur 
Implementierung, die in der Literatur beschriebenen Mechanismen eingesetzt 
werden können oder sich die erarbeiteten Ansätze übertragen lassen. 
Durch die Durchführung dieser Analyse soll ein Beitrag dazu geleistet 
werden, einen ethisch und moralisch verantwortungsbewussten Umgang mit 
bildgenerierenden KIs zu fördern.