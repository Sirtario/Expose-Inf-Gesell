\chapter{Methodik}
Die vorliegende Arbeit soll die Begriffe bildgenerierende KI (IGAI) und Moral in Bezug auf diese KIs betrachten und 
die Relevanz der Moral im Umgang mit KI erläutern. Hierbei sollen die genannten Begriffe zunächst definiert und im 
Anschluss diskutiert werden. Des Weiteren wird in dieser Arbeit untersucht, an welchen konkreten Punkten im Lebenszyklus 
einer IGAI die in der Literatur beschriebenen Mechanismen eingesetzt werden können oder sich die erarbeiteten Ansätze übertragen 
lassen. Mithilfe dieser Analyse soll ein Beitrag dazu geleistet werden, einen ethisch und moralisch verantwortungsbewussten 
Umgang mit bildgenerierenden KIs zu fördern.