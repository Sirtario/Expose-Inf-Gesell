\documentclass[12pt]{article}

\usepackage[utf8]{inputenc}
\usepackage{ngerman}
\usepackage{graphicx}
\usepackage{hyperref}

\title{Stand der Forschung \\[1ex] \large Thema: Welche Methodiken sind möglich bzw. werden angewandt, um die Erzeugung ethisch fragwürdiger Inhalte durch IGAI zu verhindern?}
\date{16.04.2023}
\author{Ronja Drechsler, \and Dominic Fitter, \and Michel Hecker, \and Khaldon Kassem, \and Phillip Eckstein}

\begin{document}

\maketitle
\tableofcontents
\section{Einleitung}
In den letzten Jahren haben Fortschritte in der Künstlichen Intelligenz (KI) dazu geführt, dass Bildgenerierende KI-Systeme immer 
leistungsfähiger geworden sind. Diese Systeme können nun hochauflösende Bilder in beispiellosem Detail generieren und haben das Potenzial,
 in vielen Branchen eingesetzt zu werden, einschließlich der Unterhaltungsindustrie, des Einzelhandels und des Gesundheitswesens. 
 Allerdings haben diese Fortschritte auch einige Bedenken hervorgerufen, insbesondere in Bezug auf die ethischen Implikationen, 
 die damit verbunden sind.
 in wichtiger Aspekt, der bei der Verwendung von bildgenerierenden KI-Systemen berücksichtigt werden muss, ist die Möglichkeit, 
 dass ethisch fragwürdige Inhalte generiert werden können. Es gibt ein wachsendes Bewusstsein für die Notwendigkeit, 
 sicherzustellen, dass KI-Systeme keine rassistischen, sexistischen oder anderweitig diskriminierenden Inhalte generieren. Diese 
 Herausforderung stellt eine wichtige Forschungsfrage dar, die untersucht werden muss, um sicherzustellen, dass die Verwendung von
  bildgenerierenden KI-Systemen ethisch vertretbar bleibt.

 \subsection{Motivation und Relevanz}
 Die Entwicklung von AI-Kunst reicht zurück bis in die 1960er Jahre\cite{Garcia}, aber erst in den letzten Jahren wurden 
 diese KIs durch die Veröffentlichungen von bahnbrechenden Projekten wie DALL-E (01/21), Midjourney (03/22) oder Stable Diffusion 
 (08/22) der Allgemeinheit zugänglich gemacht. Auslotungen der Leistungsfähigkeit dieser KIs zeigen jedoch auch auf, 
 dass sie sexistische oder rassistische Verzerrungen aufweisen \cite{Schmidt}und für moralisch fragwürdige Zwecke missbraucht werden 
 können, z. B. bei der Erstellung von Deepfakes, pornografischen oder gewalttätigen Inhalten.\cite{Hadero}

 Angesichts dessen haben u. a. die Entwickler von Midjourney reagiert und das Verwenden von Wörtern mit Bezug zur 
 Thematik menschlicher Fortpflanzungssysteme verboten \cite{Heikkilae} und ihre kostenlose Testversion eingestellt. 
 \cite{NelsonMidjourney} Doch es stellt sich die 
 Frage, wie Anbieter und Entwickler solcher KIs gegen diese schädliche Verwendung ihrer Technologie vorgehen können.
 
 Die vorliegende Arbeit wird sich mit dieser Frage auseinandersetzen und untersuchen, welche Möglichkeiten es gibt, um KIs vor 
 Missbrauch zu schützen. Sie wird diskutieren, wie die Anbieter und Entwickler dieser Technologie verantwortungsbewusst 
 handeln können, um sicherzustellen, dass ihre KIs nur für ethisch vertretbare Zwecke eingesetzt werden. Das Ziel ist es, 
 Wege aufzuzeigen, wie KI-Entwickler dazu beitragen können, die positiven Auswirkungen der Technologie zu maximieren, 
 während sie gleichzeitig die negativen Auswirkungen minimieren.
 \subsection{Forschungsfrage}
 In dieser Arbeit soll die Frage geklärt werden welche Methodiken möglich sind bzw. angewandt werden, um die Erzeugung ethisch 
 fragwürdiger Inhalte durch IGAI zu verhindern?
 \section{Stand der Forschung}
Nachfolgend wird der Stand der Forschung zum Thema Möglichkeiten zur Verhinderung 
des Missbrauchs bildergenerierender Kis zur Erzeugung ethisch fragwürdiger Inhalte beleuchtet. Eine Bestandaufnahme 
mit der Forschungsfrage, wie in der Praxis verhindert wird, dass IGAI zur Erzeugung ethisch fragwürdiger 
Inhalte missbraucht werden erfolgt. Dafür soll zunächst auf die eigentliche Recherche und die daraus mündenden 
Ergebnisse eingegangen werden, bevor die für die Recherche angewendeten KI-Werkzeuge genannt und deren 
Nutzen diskutiert wird. Zuletzt wird ein Ausblick gegeben, wie die Erarbeitung des Forschungsstands noch 
verfeinert und abgeschlossen werden kann.
\subsection{Recherche}
Für die Recherche fanden folgende Datenbanken und Webseiten Anwendung:
\begin{itemize}
    \item Google Scholar
    \item link.springer.org
    \item Semantic Scholar
    \item ACM
    \item Regensburger Katalog
    \item Katalog der Universitätsbibliothek Leipzig
    \item Katalog der Deutschen Nationalbibliothek
    \item Bibliothekskatalog der Westsächsischen Hochschule Zwickau
    \item IEEE
    \item IBM    
\end{itemize}

Zudem dienten KI-gestützte Recherchetools wie Elicit und ChatGPT-3 sowie WikiCFP der groben 
Orientierung innerhalb des Themas und inspirierten zur Nutzung einiger der oben genannten 
Datenbanken wie Semantic Scholar. Dabei wurden Fragenstellungen wie die folgenden genutzt:
How can the generation of unethical content through picture generating ai be prevented?
How to develop image generating AI so that users can not create unethical imges with it
Methods to prevent the generation of unethical content in AI
Eine Einschätzung der Dienlichkeit der verwendeten Werkzeuge folgt in Kapitel 4.

\subsection{Auswahl der Quellen}
Bei der Suche nach Quellen wurde eine Vielzahl an verschiedenen Datenbanken verwendet, um ein möglichst großes 
Spektrum an Material abzudecken. Dabei wurden verschiedene Arten von Fragestellungen verwendet. Zum einen wurden 
Fragestellungen zu der Forschungsfrage im Allgemeinen formuliert und zum anderen konkrete Fragestellungen zu einzelnen 
Teilbereichen des Themas. Bei der Auswahl wurde sich lediglich auf frei zugängliche Quellen bezogen, da ansonsten der 
finanzielle und zeitliche Rahmen dieser Arbeit überschritten worden wäre. Für die Entscheidung, ob ein Werk für die 
Verwendung in der vorliegenden Arbeit geeignet ist, wurde anhand des Abstracts überprüft, ob das Paper Aspekte der 
Forschungsfrage thematisiert oder ob die Ergebnisse des Papers genutzt werden können, um die Forschungsfrage zu beantworten. 
Je nach Datenbank standen zusätzlich Werkzeuge zur Verfügung, die es ermöglichen die Arbeiten qualitativ zu bewerten. 
Wenn diese Werkzeuge zur Verfügung standen, wurden vorzugsweise Paper mit einer höheren Bewertung verwendet. Aufgrund der 
Neuheit der Thematik und beschränkten Datenlage wurden allerdings nur wenige Werke aufgrund niedriger Bewertung außer Acht gelassen.

\subsection{Forschungsstand}
Werke wie \cite{Salminen} zeigen, dass KI nicht ethisch ist, sondern Ergebnisse auf Basis ihres Datensatzes liefert,
und Bias schnell zu diskriminierenden Verteilungen von Ergebnismengen führen können. \cite{Jobin} und \cite{Partadiredja}
sind Beispiele für Werke, die bereits vorhandene Richtlinien und ähnliche Werke zu ethischen Festlegungen bezüglich KI 
aufgreifen und darlegen, wobei \cite{Partadiredja} auch Unterschiede zwischen von KI und von Menschen generierte 
Mediacontent einschließlich Bildern herausarbeitet und dabei insbesondere auf ethische Implikationen eingeht. 
Es gibt solche weisenden, jedoch nicht gesetzlich bindenden Vorschriften also, jedoch demonstrieren Werke wie 
\cite{Ayling}, dass diese bei der Entwicklung und dem Einsatz von KI noch nicht ausreichend praktische Anwendung 
finden. \cite{Ayling} zeigt hierzu explizit Defizite aktueller Werkzeuge für Audits und Risikobewertungen bezüglich 
ethischer Rahmenwerke und Grundsätze in der KI auf, die in Zukunft berücksichtigt werden sollten. 
Prinzipiell gibt es bereits Arbeiten, deren Ergebnisse die Umsetzung verschiedener Richtlinien fördern können, so legt 
\cite{EUCommision} Anforderungen an KI in der EU fest. \cite{Jobin} und \cite{Hagendorff} definieren Strategien 
für die Implementierung von Richtlinien für ethische Prinzipien für KI. \cite{Stahl} diskutiert die Ethik von KI und Robotik 
und liefert Einblicke, wie ethische Prinzipien im Allgemeinen in der Praxis angewendet werden können. \cite{Srinivasan} bietet 
eine Zusammenstellung von Bias in verschiedenen Phasen des KI-Prozesses und gibt Empfehlungen für bewährte Verfahren und 
Richtlinien zur Identifizierung und Minderung von Verzerrungen in KI-Systemen für Entwickler von maschinellem Lernen. \cite{Jameel} 
geht auf Grundlage einer Vorstellung verschiedener KI-Modelle auf bestimmte ethische Probleme ein und zeigt Wege auf, 
Daten von ethisch hoher Qualität zu erhalten.
Zudem geben einige Unternehmen Überblick über ihre Ansätze zur Verminderung von Verzerrungen bei der Benutzung von KI, 
beispielsweise IBM über \cite{Hobson}. In diesem Werk wird auch anhand von Prozessen dargelegt, wie ethisch korrekt mit KI 
umgegangen werden kann.
Eine Variante, bestimmte Inhalte zu blockieren, besteht in der Filterung. Einen solche Filter für Bilder beschreibt \cite{Zheng}.
Die beschriebene Software kann Haut auf Bildern erkennen und darüber z. B. nicht kinder- und jugendfreie Inhalte und Symbole herausfiltern.
Zusammenfassend lässt sich also sagen, dass die Problematik bei der Umsetzung ethischer Prinzipien in der KI erkannt und
durch Richtlinien u. ä. Anleitungen gegeben werden, wie diese geschehen kann. Außerdem gibt es Untersuchungen 
zur Anforderungsanalyse und Vorschläge zur Implementierung für diese Umsetzung, allerdings keinen zusammenfassenden 
Überblick, was denn tatsächlich praktisch getan wird, um Missbrauch bildergenerierender Kis zur Erzeugung ethisch 
fragwürdiger Inhalte zu verhindern.
Nennung und Einschätzung der verwendeten KI-Werkzeuge

Wie bereits in Kapitel 2 erwähnt, wurden für die Recherche verschiedene KI-Werkzeuge verwendet, deren Nutzung im Folgenden 
eingeschätzt und die Nützlichkeit dessen diskutiert werden soll. 
ChatGPT3 ist ein Sprachmodell und dient somit der Fomulierung sprachlich korrekter Texte, kann jedoch beispielsweise keine Paper 
zusammenfassen, sondern gibt nur anhand des Titels jeweils ähnlich klingende Zusammenfassungen, ohne sich dabei auf die tatsächlichen 
Inhalte des Papers berufen zu können. Somit eignet sich höchstens, um die betrachtete Disziplin erforschende Autoren zu finden 
und auf dieser Basis mit anderen Werkzeugen weiterzurecherchieren. Solche Unterstützung liefern allerdings auch wissenschaftlich
etabliertere Webseiten wie WikiCFP. Zudem könnte ChatGPT3 von Wissenschaftlern, die in der Sprache, in der sie eine wissenschaftliche 
Arbeit schreiben, nicht ausreichend versiert sind, als Unterstützung bei der Formulierung eines wohlklingenden Texts verwendet werden.
Neben dem reinen ChatGPT3 wurde Bing mit ChatGPT3 verwendet. Dies bietet gegenüber der oben beschriebenen Variante den Vorteil, 
dass Bing das Internet durchsucht und somit inhaltlich passendere Antworten liefern kann. Der Nachteil besteht darin, dass nur das 
Internet beipielsweise nach dem Papertitel und „summary“ durchsucht wird und eines der ersten Suchergebnisse in wohlformulierter 
Sprache widergegeben wird, weshalb auch hier keine wirklich brauchbaren Inhalte erzeugt werden.
Generell muss die Benutzung eines Werkzeugs geübt werden, um die Möglichkeiten dessen in guter Qualität auszuschöpfen. Demzufolge 
müsste länger und intensiver mit diesem Werkzeug gearbeitet werden, um eventuell nützlichere Ergebnisse zu erzielen als diese. Die 
oben genannte Einschätzung beruht nur auf einwöchige Nutzung ohne nennenswerte Vorkenntnisse und ist demnach nicht repräsentativ für 
das tatsächliche Potenzial der genutzten KI-Werkzeuge.
Dasselbe gilt für Elicit. Dieses Werkzeug ist durch die Größe seiner Datenbank beschränkt, innerhalb dieser jedoch hilfreich, um 
einen Überblick über Werke zu erlangen, deren Autoren nicht auf den hiesigen Konferenzen vertreten und folglich mithilfe der 
klassischen Recherche über Konferenzen und deren Paper und Teilnehmer nicht auffindbar wären. Mithilfe geeigneter Fragen kann 
auch Zeit bei der Recherche gespart werden, indem die Zusammenfassung der ersten vier Paper genutzt wird. Diese sollte jedoch 
nicht nur, wenn die eigentlich intendierte Frage nicht direkt beantwortet wird, überprüft werden, da sie sehr kurz und nicht 
immer vollständig zutreffend ist.

Die betrachteten Arbeiten lassen sich somit wie folgt inhaltlich gruppieren: Es gibt wissenschaftliche Arbeiten zur Analyse und Filterung
bestimmter Inputs/Outputs wie Haut und Symbole aus Bildern (Zheng 2004) oder unmoralische Wörter oder Phrasen [Shah 2022]. Des weiteren 
werden Richtlinien zum ethischen Einsatz von KI zusammenstellt oder definiert und diskutiert [Ayling 2021] [Srinivasan, Chander 2021] 
[Jameel 2020] [Hagendorff 2020] [Jobin 2019] [Unity Technologies 2018] [European Comission 2018] [World Economic Forum 2018] [Müller 2021].
Neben diesen Richtlinien werden Frameworks für KI entwickelt [Huang et al. 2022] [Müller 2021]. Es werden Probleme von Kis ermittelt [Ayling 2021], 
beispielsweise Bias [Salminen et al. 2020] [Jameel 2020], Kopien [Somepalli et al. 2022], wobei u. a. Bias auch zu ethisch 
fragwürdigen Resultaten führen kann [Zuber 2022]. Lösungsansätze für diese Probleme werden allgemein für KI-spezifische Probleme ohne 
Konzentration auf ein bestimmtes [Ayling 2021] [Avelar 2022], aber auch konkret für bestimmte Problemkategorien wie Bias [Srinivasan 2021] 
[Jameel 2020] untersucht. Technikfolgen und Regulierungsfragen für verschiedene von KI beeinflusste Kontexte wie Politik und Wirtschaft
werden analysiert [Pawelec, Bieß 2021].

\subsection{Methoden der Quellen}
\chapter{Methodik}

Die genutzten Quellen verwenden verschiedene Methodiken, um ihre Ergebnisse zu erreichen. Es wurden Experimente angefertigt wie in \cite[Blocking objectionable images]{Zheng}. Hier wurde zum Beispiel ein Bild-Filter gegen pornografische Inhalte erstellt. Minikken \cite{Minkkinen} zeigt eine Nutzweranalyse. Des Weiteren wurden verschiedene Literaturübersichten angefertigt wie in \cite[Biases in AI Systems]{Srinivasan} oder \cite[The global landscape of AI ethics guidelines]{Jobin}.
Auch in der vorliegenden Arbeit soll eine Übersicht anhand von Literatur erstellt werden. Im Unterschied zu den genannten Literaturübersichten liegt der Fokus jedoch darauf, die Mechanismen zu erfassen, die eingesetzt werden, um die unethische Verwendung von bildgenerierender KI (IGAI) zu verhindern. Als Grundlage hierfür werden zunächst bildgenerierende KI (IGAI) und Moral in Bezug auf diese KIs definiert und zur Erläuterung der
Relevanz von Moral beim Umgang mit KI diskutiert. Danach werden die Mechanismen aus der betrachteten Literatur erfasst und es wird untersucht, an wann im Lebenszyklus einer IGAI sie eingesetzt werden können oder sich die erarbeiteten Ansätze übertragen lassen.
\section{Definitionen}
Wie in der Methodik erwähnt, werden im Folgenden die Begriffe der bildgenerienden KI und der Ethik im Bezug auf dieser für diese Arbeit definiert. 
Durch diese Vorgehensweise soll vermieden werden, dass diese Thematiken bei kommenden Durchführung immer neu angesprochen werden müssen und sich die Arbeit so auf klare und einheitliche Definitionen der Begriffe stützen kann.
\subsection{Ethik}\label{subsection:ethicsdefinition}
\subsection{AI}
Im Gegensatz zu bisherigen AIs können den generative artifical intelligence (GAI) neue Daten generieren, anstatt wie bisher Daten zu analysieren oder basierend auf gegebenen Daten zu handeln. Außerdem werden ihnen zum Trainieren enorme Datenmengen, wie der komplette Inhalt von Wikipedia, zur Verfügung gestellt. Daher sind diese AIs erst mit steigender Rechenkapazität möglich geworden. GAIs modellieren Netze aus miteinander verbundenen Daten in verschiedenen multimedialen Formaten. Demnach bestünden beispielsweise Verbindungen zwischen dem Wort Panda, ein Bild von einem Panda und einem Video von einem Panda. Dadurch können GAIs ein beliebiges Eingabeformat in ein beliebiges Ausgabeformat übersetzen. Thema dieser Arbeit sind image generating artifical intelligence (IGAI) oder auch text-to-image AIs, eine Unterkategorie von GAIs. Diese AIs können nach Eingabe von natürlichsprachlichem Text Bilder erzeugen.


Um GAIs zu trainieren, werden drei Arten von Machine Learning verwendet: überwachtes Lernen, unüberwachtes Lernen und verstärktes Lernen. 
Beim überwachten Lernen werden gekennzeichnete Datensätze verwendet, um Vorhersagen zu treffen. Die gekennzeichneten Daten werden als Trainingsdaten verwendet und nach dem Ausführen des Algorithmus wird überprüft, ob die getroffene Vorhersage der Kennzeichnung entspricht.
Beim unüberwachten Lernen wird dem Algorithmus nicht gekennzeichnete Daten zugeführt. Hierbei wird nicht versucht eine Vorhersage zu treffen, sondern Schlussfolgerungen oder Zusammenhänge zwischen den Daten zu finden.
Beim verstärkten Lernen wird ein System aus Belohnung und Strafen verwendet. Macht der Algorithmus wenige Fehler ist die Belohnung groß und die Strafe klein, macht der Algorithmus viele Fehler ist die Belohnung klein und die Strafe groß. Dadurch kann das Finden des optimalen Verhaltens automatisiert werden.

\section{Durchführung}
Verschiedene Formen der Voreignenommenheit von AI lassen sich durch unterschiedliche Methoden verhindern. Diese lassen sich allgemein in 
Technische und Nicht-Technische Methoden unterteilen. Nachfolgen werden verschiedene Methoden erläutert.
\subsection{Nicht-Technische Methoden}
Nicht-Technische Methoden sind solche, die nicht primär auf Technische Lösungen setzen.

Eine Möglichekeit des Vermeidens Von Voreignenommenheit ist der Einbezug von möglichst vieler Anteilhaber aus verschiedenen Gruppierungen. 
Ziel des Einbezugs ist es, möglichst viele unterschiedliche Perspektiven zu erhalten, was dabei helfen soll mögliche Voreignenommenheiten zu erkennen und vermeiden.

\subsection{Technische Methoden}
Technische Ansätze sind diese, die von den Entwicklern der IGAI sowohl während als auch nach der Entwicklungsphase eingesetzt werden können, beispielsweise in Form von Datenaufbereitung und -manipulation oder Filter. 
Aufgrund dessen werden in der weiteren Durchführung diese Ansätze danach gruppiert, ob diese in der Entwicklungsphase der IGAI oder in deren Anwendungsphase anzuwenden sind.
\subsubsection{Vermeidung von Datenverzerrungen im Trainingsdatensatz}
Bereits bei der Zusammenstellung des Trainingsdatensatzes einer KI kann es zu Verzerrungen kommen, welche sich im späteren Verlauf der Entwicklung kaum mehr beheben lassen.
Diese beinhalten Stichprobenverzerrungen, Messwertverzerrungen, Kennzeichnungsverzerrungen, oder Verzerrung durch Einseitigkeit \cite{Srinivasan}.

Beispiele für Stichprobenverzerrungen können sein, dass zu wenig Daten in die Datensätze aufgenommen werden und dadurch nicht realitätsnah sind \cite{Srinivasan}.
Auch ist es möglich, dass die Daten systematisch einseitig sind. Ein Gesichtserkennungsmodell, welches hauptsächlich an hellhäutigen Gesichtern trainiert wurde,
kann so beispielweise bei dunkleren Hauttönen schlechter abschneiden, oder bestimmte Gesichtsformen und -farben gar nicht als solche wahrnehmen \cite{Srinivasan}.
Ebenfalls ist eine systematisch Messwertverzerrung möglich, etwa wenn Bilder, welche mit einer defekten Kamera aufgenommen worden, in die Trainingsdaten einbezogen werden.
Es kann auch vorkommen, dass Daten falsch oder unzuverlässig gekennzeichnet werden. so kann es vorkommen. Eine Wiese kann als \textit{Rasen}, \textit{Gras}, \textit{Wiese} oder \textit{Weide} gekennzeichnet werden und erhält damit vier Bezeichnungen für das gleiche Objekt existieren. Dies kann sich insgesamt negativ auf die Genauigkeit des Modells auswirken \cite{Srinivasan}.

Diese Verzerrung lassen sich vermeiden, indem man den Prozess der Datensammlung von neutraler Seite überwacht und verifiziert.
Bei dieser Überwachung sollten mindestens die genannten Punkte Beachtung finden: 
\begin{itemize}
    \item Die Daten müssen die Realität widerspiegeln und nicht nur Randgruppen oder Mehrheiten darstellen.
    \item Die Anzahl der Daten muss ausreichen um diese Realitätswiederspiegelung gewährleisten zu können.
    \item Die Kennzeichnung der Daten muss auf Voreingenommenheit und Redundanz geprüft werden.
    \item Die Daten müssen qualitativ hochwertig, mindestens aber für den Anwendungsfall ausreichend sein.
\end{itemize}

\subsubsection{Vermeidung von Verzerrungen in der Anforderungsformulierungen}
Bei den Definitionen einer Aufgabenstellung und den Anforderungen eines Modell kann es zu Verzerrungen kommen.

Diese Verzerrung kann einerseits durch die Formulierung der Aufgabenstellung entstehen, oder durch die Art der verwendeten Daten um das Ergebnis des Models zu erhalten \cite{Srinivasan}.
Ein Problem bei der Formulierung der Anforderungen an einen Algorithmus kann beispielsweise sein, Daten wie Geschlecht, Hautfarbe oder ethnische Zugehörigkeit in die 
Entscheidung einfließen zu lassen, ob einer Person ein Kredit gewährt wird. Hierdurch können sexistische oder rassistische Diskriminierungen bei der Entscheidung stattfinden \cite{Srinivasan}.

Bei der Formulierung der Anforderungen des Systems sollte eine Reihe an Aspekten berücksichtigt und durch einen neutralen Beobachter verifiziert werden:
\begin{itemize}
    \item Bei dem Treffen einer Entscheidung des Modells dürfen keine Aspekte berücksichtigt werden, die für ein erfolgreiches Ausführen des Algorithmus nicht nötig sind.
    \item Die Formulierung muss neutral gestaltet sein, das heißt durch die Formulierung an sich keine voreingenommenen Schlüsse ziehen lässt.
\end{itemize}

\subsubsection{Vermeidung von Verzerrungen bei der Modellbewertung}
Neben der Möglichkeit, dass die Trainingsdaten oder der Algorithmus Verzerrungen enthalten kann es auch sein, dass die Evaluation des Modells Verzerrungen aufweist.
Diese können durch menschliches Verhalten der Tester, oder durch schlecht entworfene Test-Frameworks auftreten \cite{Srinivasan}.

Ein menschliche Bewerter eines Modells können in veschiedenen Aspekten kritisch handeln. Sie können durch eigene Einstellungen und Erfahrungen auch unabsichtlich beeinflußt sein \cite{Srinivasan}.
Gerade wenn ein Bewerter direkt am Model beteiligt ist kann es zum so genannten Bestätigungsfehler, oder 	\glqq Peak-End-Effect\grqq{} kommen \cite{Srinivasan}.
Ein Bestätigungsfehler ist der Neigung durch subjektive Wahrnemung eine Erwartung bestätigt zu sehen \cite{Srinivasan}. 
Beim \glqq Peak-End-Effect\grqq{} wird eine Bewertung anhand von einzelnen Höhepunkten und dem letzten Eindruck geprägt \cite{Srinivasan}.
Ebenfalls kann ein entworfenes Test-Framework fehlerhaft sein, wenn darin zum Beispiel zu wenig getestet wird, oder mit nicht geeigneten Testdaten gearbeitet wird \cite{Srinivasan}.

Um ein Modell ausreichend zu testen sollten einige Punkte berücksichtigt werden:
\begin{itemize}
    \item Wenn Menschen das Modell bewerten sollten diese nicht direkt an der Entwicklung beteiligt sein, um so Neutralität zu wahren.
    \item Wenn das Modell durch Menschen evaluiert wird, sollte diese Evalutaion von unteschiedlichen Personen mit unterschiedlichen Hintergründen durchgeführt werden.
    \item Wird für die Evaluation ein Test-Framework verwendet, muss dieses mindestens eine für den Anwedungsfall annehmbare Menge an voneinander unabhängigen Tests durchführen.
    \item Wird für die Evaluation ein Test-Framework verwendet, muss dieses weitreichend auf systematische Fehler überprüft werden.
\end{itemize}


\subsubsection{Verwendung von Datenerweiterung (Data Augmentation)}
\subsubsection{Verwendung von überwachtem Lernen (Supervised Learning)}
Die Trainingsdatensätze werden im Voraus bewertet und sortiert, bevor der Algorithmus durchlaufen wird. Nachdem der Algorithmus durchlaufen ist, wird das Ergebnis mit der Bewertung verglichen. 
Das Modell kann aus diesem Vergleich lernen, welches Verhalten oder Ergebnis richtig und welches falsch ist.

Nach diesem Prinzip könnte man für IGAIs Datensätze erstellen, welche aus 2-Tupeln bestehen: einem Text anhand dem ein Bild erzeugt werden soll und einem dazu passenden moralisch unbedenklichen Bild. 
Das Modell kann so anhand von positiven Beispielen lernen, welche Inhalte es erzeugen darf.

Weiter wäre auch das Gegenteil möglich: Ein geordnetes Paar an Texten und moralisch verwerflichen Bildern, anhand derer die KI lernen kann, was sie auf keinen Fall generieren darf.
\subsubsection{Verwendung von verstärktem Lernen (Reinforcement Learning)}
Wie in \ref{subsection:ethicsdefinition} erklärt, wird beim verstärkten Lernen ein Belohnungs-Bestrafungs-System eingesetzt. Dieses könnte im Fall von IGAIs daraus bestehen, einen zweiten Agenten zu verwenden, welcher die erzeugten Inhalte auf moralische Sauberkeit und dementsprechend den ersten Agenten, also die IGAI, bestraft oder belohnt.
\subsubsection{Verwendung von kontradiktorischen Training (adversarial training)}



\section{Fazit}
\bibliographystyle{ieeetr}
\bibliography{Sources.bib}
\end{document}