\documentclass[12pt]{article}

\usepackage[utf8]{inputenc}
\usepackage{ngerman}
\usepackage{graphicx}
\usepackage{hyperref}

\title{Stand der Forschung \\[1ex] \large Thema: Welche Methodiken sind möglich bzw. werden angewandt, um die Erzeugung ethisch fragwürdiger Inhalte durch IGAI zu verhindern?}
\date{16.04.2023}
\author{Ronja Drechsler, \and Dominic Fitter, \and Michel Hecker, \and Khaldon Kassem, \and Phillip Eckstein}

\begin{document}

\maketitle
\tableofcontents
\section{Einleitung}
In den letzten Jahren haben Fortschritte in der Künstlichen Intelligenz (KI) dazu geführt, dass Bildgenerierende KI-Systeme immer 
leistungsfähiger geworden sind. Diese Systeme können nun hochauflösende Bilder in beispiellosem Detail generieren und haben das Potenzial,
 in vielen Branchen eingesetzt zu werden, einschließlich der Unterhaltungsindustrie, des Einzelhandels und des Gesundheitswesens. 
 Allerdings haben diese Fortschritte auch einige Bedenken hervorgerufen, insbesondere in Bezug auf die ethischen Implikationen, 
 die damit verbunden sind.
 in wichtiger Aspekt, der bei der Verwendung von bildgenerierenden KI-Systemen berücksichtigt werden muss, ist die Möglichkeit, 
 dass ethisch fragwürdige Inhalte generiert werden können. Es gibt ein wachsendes Bewusstsein für die Notwendigkeit, 
 sicherzustellen, dass KI-Systeme keine rassistischen, sexistischen oder anderweitig diskriminierenden Inhalte generieren. Diese 
 Herausforderung stellt eine wichtige Forschungsfrage dar, die untersucht werden muss, um sicherzustellen, dass die Verwendung von
  bildgenerierenden KI-Systemen ethisch vertretbar bleibt.

 \subsection{Motivation und Relevanz}
 Die Entwicklung von AI-Kunst reicht zurück bis in die 1960er Jahre\cite{Garcia}, aber erst in den letzten Jahren wurden 
 diese KIs durch die Veröffentlichungen von bahnbrechenden Projekten wie DALL-E (01/21), Midjourney (03/22) oder Stable Diffusion 
 (08/22) der Allgemeinheit zugänglich gemacht. Auslotungen der Leistungsfähigkeit dieser KIs zeigen jedoch auch auf, 
 dass sie sexistische oder rassistische Verzerrungen aufweisen \cite{Schmidt}und für moralisch fragwürdige Zwecke missbraucht werden 
 können, z. B. bei der Erstellung von Deepfakes, pornografischen oder gewalttätigen Inhalten.\cite{Hadero}

 Angesichts dessen haben u. a. die Entwickler von Midjourney reagiert und das Verwenden von Wörtern mit Bezug zur 
 Thematik menschlicher Fortpflanzungssysteme verboten \cite{Heikkilae} und ihre kostenlose Testversion eingestellt. 
 \cite{NelsonMidjourney} Doch es stellt sich die 
 Frage, wie Anbieter und Entwickler solcher KIs gegen diese schädliche Verwendung ihrer Technologie vorgehen können.
 
 Die vorliegende Arbeit wird sich mit dieser Frage auseinandersetzen und untersuchen, welche Möglichkeiten es gibt, um KIs vor 
 Missbrauch zu schützen. Sie wird diskutieren, wie die Anbieter und Entwickler dieser Technologie verantwortungsbewusst 
 handeln können, um sicherzustellen, dass ihre KIs nur für ethisch vertretbare Zwecke eingesetzt werden. Das Ziel ist es, 
 Wege aufzuzeigen, wie KI-Entwickler dazu beitragen können, die positiven Auswirkungen der Technologie zu maximieren, 
 während sie gleichzeitig die negativen Auswirkungen minimieren.
 \subsection{Forschungsfrage}
 In dieser Arbeit soll die Frage geklärt werden welche Methodiken möglich sind bzw. angewandt werden, um die Erzeugung ethisch 
 fragwürdiger Inhalte durch IGAI zu verhindern?
 \chapter{Stand der Forschung}

Nachfolgend wird der Stand der Forschung zum Thema Möglichkeiten zur Verhinderung 
des Missbrauchs bildergenerierender Kis zur Erzeugung ethisch fragwürdiger Inhalte mit Fokus darauf beleuchtet, wie in der Praxis verhindert wird, dass IGAI zur Erzeugung ethisch fragwürdiger Inhalte missbraucht werden. Anschließend wird die Nützlichkeit der für die Recherche verwendeten KI-Anwendungen kurz diskutiert.

Werke wie \cite{Salminen} zeigen, dass KI nicht ethisch ist, sondern Ergebnisse auf Basis ihres Datensatzes liefert,
und Bias schnell zu diskriminierenden Verteilungen von Ergebnismengen führen können. \cite{Jobin} und \cite{Partadiredja}
sind Beispiele für Werke, die bereits vorhandene Richtlinien und ähnliche Werke zu ethischen Festlegungen bezüglich KI 
aufgreifen und darlegen, wobei \cite{Partadiredja} auch Unterschiede zwischen von KI und von Menschen generierte 
Mediacontent einschließlich Bildern herausarbeitet und dabei insbesondere auf ethische Implikationen eingeht. 
Es gibt solche weisenden, jedoch nicht gesetzlich bindenden Vorschriften also, jedoch demonstrieren Werke wie 
\cite{Ayling}, dass diese bei der Entwicklung und dem Einsatz von KI noch nicht ausreichend praktische Anwendung 
finden. \cite{Ayling} zeigt hierzu explizit Defizite aktueller Werkzeuge für Audits und Risikobewertungen bezüglich 
ethischer Rahmenwerke und Grundsätze in der KI auf, die in Zukunft berücksichtigt werden sollten. 
Prinzipiell gibt es bereits Arbeiten, deren Ergebnisse die Umsetzung verschiedener Richtlinien fördern können, so legt 
\cite{EUCommision} Anforderungen an KI in der EU fest. \cite{Jobin} und \cite{Hagendorff} definieren Strategien 
für die Implementierung von Richtlinien für ethische Prinzipien für KI. \cite{Stahl} diskutiert die Ethik von KI und Robotik 
und liefert Einblicke, wie ethische Prinzipien im Allgemeinen in der Praxis angewendet werden können. \cite{Srinivasan} bietet 
eine Zusammenstellung von Bias in verschiedenen Phasen des KI-Prozesses und gibt Empfehlungen für bewährte Verfahren und 
Richtlinien zur Identifizierung und Minderung von Verzerrungen in KI-Systemen für Entwickler von maschinellem Lernen. \cite{Jameel} 
geht auf Grundlage einer Vorstellung verschiedener KI-Modelle auf bestimmte ethische Probleme ein und zeigt Wege auf, 
Daten von ethisch hoher Qualität zu erhalten.
Zudem geben einige Unternehmen Überblick über ihre Ansätze zur Verminderung von Verzerrungen bei der Benutzung von KI, 
beispielsweise IBM über \cite{Hobson}. In diesem Werk wird auch anhand von Prozessen dargelegt, wie ethisch korrekt mit KI 
umgegangen werden kann.
Eine Variante, bestimmte Inhalte zu blockieren, besteht in der Filterung. Einen solche Filter für Bilder beschreibt \cite{Zheng}.
Die beschriebene Software kann Haut auf Bildern erkennen und darüber z. B. nicht kinder- und jugendfreie Inhalte und Symbole herausfiltern.
Zusammenfassend lässt sich also sagen, dass die Problematik bei der Umsetzung ethischer Prinzipien in der KI erkannt und
durch Richtlinien u. ä. Anleitungen gegeben werden, wie diese geschehen kann. Außerdem gibt es Untersuchungen 
zur Anforderungsanalyse und Vorschläge zur Implementierung für diese Umsetzung, allerdings keinen zusammenfassenden 
Überblick, was denn tatsächlich praktisch getan wird, um Missbrauch bildergenerierender Kis zur Erzeugung ethisch 
fragwürdiger Inhalte zu verhindern.
Nennung und Einschätzung der verwendeten KI-Werkzeuge

Die betrachteten Arbeiten lassen sich somit wie folgt inhaltlich gruppieren: Es gibt wissenschaftliche Arbeiten zur Analyse und Filterung
bestimmter Inputs/Outputs wie Haut und Symbole aus Bildern \cite{Zheng} oder unmoralische Wörter oder Phrasen \cite{Shah}. Des weiteren 
werden Richtlinien zum ethischen Einsatz von KI zusammenstellt oder definiert und diskutiert \cite{Ayling} \cite{Srinivasan} 
\cite{Jameel} \cite{Hagendorff} \cite{Jobin} \cite{Unity} \cite{EUCommision}[World Economic Forum 2018] \cite{Mueller}.
Neben diesen Richtlinien werden Frameworks für KI entwickelt [Huang et al. 2022] \cite{Mueller}. Es werden Probleme von Kis ermittelt \cite{Ayling}, 
beispielsweise Bias \cite{Salminen} \cite{Jameel}, Kopien \cite{Somepalli}, wobei u. a. Bias auch zu ethisch 
fragwürdigen Resultaten führen kann [Zuber 2022]. Lösungsansätze für diese Probleme werden allgemein für KI-spezifische Probleme ohne 
Konzentration auf ein bestimmtes \cite{Ayling} \cite{Avelar}, aber auch konkret für bestimmte Problemkategorien wie Bias \cite{Srinivasan}
\cite{Jameel}untersucht. Technikfolgen und Regulierungsfragen für verschiedene von KI beeinflusste Kontexte wie Politik und Wirtschaft
werden analysiert \cite{Pawelec}.

Für die Recherche wurden verschiedene KI-Werkzeuge verwendet, deren Nutzung im Folgenden 
eingeschätzt und die Nützlichkeit dessen diskutiert werden soll. 
ChatGPT3 ist ein Sprachmodell und dient somit der Fomulierung sprachlich korrekter Texte, kann jedoch beispielsweise keine Paper 
zusammenfassen, sondern gibt nur anhand des Titels jeweils ähnlich klingende Zusammenfassungen, ohne sich dabei auf die tatsächlichen 
Inhalte des Papers berufen zu können. Somit eignet sich höchstens, um die betrachtete Disziplin erforschende Autoren zu finden 
und auf dieser Basis mit anderen Werkzeugen weiterzurecherchieren. Solche Unterstützung liefern allerdings auch wissenschaftlich
etabliertere Webseiten wie WikiCFP. Zudem könnte ChatGPT3 von Wissenschaftlern, die in der Sprache, in der sie eine wissenschaftliche 
Arbeit schreiben, nicht ausreichend versiert sind, als Unterstützung bei der Formulierung eines wohlklingenden Texts verwendet werden.
Neben dem reinen ChatGPT3 wurde Bing mit ChatGPT3 verwendet. Dies bietet gegenüber der oben beschriebenen Variante den Vorteil, 
dass Bing das Internet durchsucht und somit inhaltlich passendere Antworten liefern kann. Der Nachteil besteht darin, dass nur das 
Internet beipielsweise nach dem Papertitel und „summary“ durchsucht wird und eines der ersten Suchergebnisse in wohlformulierter 
Sprache widergegeben wird, weshalb auch hier keine wirklich brauchbaren Inhalte erzeugt werden.
Generell muss die Benutzung eines Werkzeugs geübt werden, um die Möglichkeiten dessen in guter Qualität auszuschöpfen. Demzufolge 
müsste länger und intensiver mit diesem Werkzeug gearbeitet werden, um eventuell nützlichere Ergebnisse zu erzielen als diese. Die 
oben genannte Einschätzung beruht nur auf einwöchige Nutzung ohne nennenswerte Vorkenntnisse und ist demnach nicht repräsentativ für 
das tatsächliche Potenzial der genutzten KI-Werkzeuge.
Dasselbe gilt für Elicit. Dieses Werkzeug ist durch die Größe seiner Datenbank beschränkt, innerhalb dieser jedoch hilfreich, um 
einen Überblick über Werke zu erlangen, deren Autoren nicht auf den hiesigen Konferenzen vertreten und folglich mithilfe der 
klassischen Recherche über Konferenzen und deren Paper und Teilnehmer nicht auffindbar wären. Mithilfe geeigneter Fragen kann 
auch Zeit bei der Recherche gespart werden, indem die Zusammenfassung der ersten vier Paper genutzt wird. Diese sollte jedoch 
nicht nur, wenn die eigentlich intendierte Frage nicht direkt beantwortet wird, überprüft werden, da sie sehr kurz und nicht 
immer vollständig zutreffend ist.
\section{Methodik}
Die vorliegende Arbeit hat zum Ziel, eine umfassende Betrachtung der Begriffe 
bildgenerierende KI (IGAI) und Moral in 
Bezug auf diese KIs zu liefern und die Relevanz der Moral im Umgang mit KI
zu erläutern. Hierbei sollen die genannten Begriffe zunächst definiert und im Anschluss diskutiert werden. 
Des Weiteren wird in dieser Arbeit untersucht, an welchen konkreten 
Punkten im Lebenszyklus einer IGAI, von der Entwicklung bis hin zur 
Implementierung, die in der Literatur beschriebenen Mechanismen eingesetzt 
werden können oder sich die erarbeiteten Ansätze übertragen lassen. 
Durch die Durchführung dieser Analyse soll ein Beitrag dazu geleistet 
werden, einen ethisch und moralisch verantwortungsbewussten Umgang mit 
bildgenerierenden KIs zu fördern.
\section{Definitionen}
Wie in der Methodik erwähnt, werden im Folgenden die Begriffe der bildgenerienden KI und der Ethik im Bezug auf dieser für diese Arbeit definiert. 
Durch diese Vorgehensweise soll vermieden werden, dass diese Thematiken bei kommenden Durchführung immer neu angesprochen werden müssen und sich die Arbeit so auf klare und einheitliche Definitionen der Begriffe stützen kann.
\subsection*{Ethik}
\subsection*{AI}
\section{Durchführung und Diskussion}
Verschiedene Formen der Voreignenommenheit von AI lassen sich durch unterschiedliche Methoden verhindern. Diese lassen sich allgemein in 
Technische und Nicht-Technische Methoden unterteilen. Nachfolgen werden verschiedene Methoden erläutert.
\subsection*{Nicht-Technische Methoden}
Nicht-Technische Methoden sind solche, die nicht primär auf Technische Lösungen setzen.

Eine Möglichekeit des Vermeidens Von Voreignenommenheit ist der Einbezug von möglichst vieler Anteilhaber aus verschiedenen Gruppierungen. 
Ziel des Einbezugs ist es, möglichst viele unterschiedliche Perspektiven zu erhalten, was dabei helfen soll mögliche Voreignenommenheiten zu erkennen und vermeiden.

\subsection*{Technische Methoden}
Technische Ansätze sind diese, die von den Entwicklern der IGAI sowohl während als auch nach der Entwicklungsphase eingesetzt werden können, beispielsweise in Form von Datenaufbereitung und -manipulation oder Filter. 
Aufgrund dessen werden in der weiteren Durchführung diese Ansätze danach gruppiert, ob diese in der Entwicklungsphase der IGAI oder in deren Anwendungsphase anzuwenden sind.
\section{Fazit}
\bibliographystyle{ieeetr}
\bibliography{Sources.bib}
\end{document}