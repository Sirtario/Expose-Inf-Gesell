\section{Stand der Forschung}
Nachfolgend wird der Stand der Forschung zum Thema Möglichkeiten zur Verhinderung 
des Missbrauchs bildergenerierender Kis zur Erzeugung ethisch fragwürdiger Inhalte beleuchtet. Eine Bestandaufnahme 
mit der Forschungsfrage, wie in der Praxis verhindert wird, dass IGAI zur Erzeugung ethisch fragwürdiger 
Inhalte missbraucht werden erfolgt. Dafür soll zunächst auf die eigentliche Recherche und die daraus mündenden 
Ergebnisse eingegangen werden, bevor die für die Recherche angewendeten KI-Werkzeuge genannt und deren 
Nutzen diskutiert wird. Zuletzt wird ein Ausblick gegeben, wie die Erarbeitung des Forschungsstands noch 
verfeinert und abgeschlossen werden kann.
\subsection{Recherche}
Für die Recherche fanden folgende Datenbanken und Webseiten Anwendung:
\begin{itemize}
    \item Google Scholar
    \item link.springer.org
    \item Semantic Scholar
    \item ACM
    \item Regensburger Katalog
    \item Katalog der Universitätsbibliothek Leipzig
    \item Katalog der Deutschen Nationalbibliothek
    \item Bibliothekskatalog der Westsächsischen Hochschule Zwickau
    \item IEEE
    \item IBM    
\end{itemize}

Zudem dienten KI-gestützte Recherchetools wie Elicit und ChatGPT-3 sowie WikiCFP der groben 
Orientierung innerhalb des Themas und inspirierten zur Nutzung einiger der oben genannten 
Datenbanken wie Semantic Scholar. Dabei wurden Fragenstellungen wie die folgenden genutzt:
How can the generation of unethical content through picture generating ai be prevented?
How to develop image generating AI so that users can not create unethical imges with it
Methods to prevent the generation of unethical content in AI
Eine Einschätzung der Dienlichkeit der verwendeten Werkzeuge folgt in Kapitel 4.

\subsection{Auswahl der Quellen}
Bei der Suche nach Quellen wurde eine Vielzahl an verschiedenen Datenbanken verwendet, um ein möglichst großes 
Spektrum an Material abzudecken. Dabei wurden verschiedene Arten von Fragestellungen verwendet. Zum einen wurden 
Fragestellungen zu der Forschungsfrage im Allgemeinen formuliert und zum anderen konkrete Fragestellungen zu einzelnen 
Teilbereichen des Themas. Bei der Auswahl wurde sich lediglich auf frei zugängliche Quellen bezogen, da ansonsten der 
finanzielle und zeitliche Rahmen dieser Arbeit überschritten worden wäre. Für die Entscheidung, ob ein Werk für die 
Verwendung in der vorliegenden Arbeit geeignet ist, wurde anhand des Abstracts überprüft, ob das Paper Aspekte der 
Forschungsfrage thematisiert oder ob die Ergebnisse des Papers genutzt werden können, um die Forschungsfrage zu beantworten. 
Je nach Datenbank standen zusätzlich Werkzeuge zur Verfügung, die es ermöglichen die Arbeiten qualitativ zu bewerten. 
Wenn diese Werkzeuge zur Verfügung standen, wurden vorzugsweise Paper mit einer höheren Bewertung verwendet. Aufgrund der 
Neuheit der Thematik und beschränkten Datenlage wurden allerdings nur wenige Werke aufgrund niedriger Bewertung außer Acht gelassen.

\subsection{Forschungsstand}
Werke wie \cite{Salminen} zeigen, dass KI nicht ethisch ist, sondern Ergebnisse auf Basis ihres Datensatzes liefert,
und Bias schnell zu diskriminierenden Verteilungen von Ergebnismengen führen können. \cite{Jobin} und \cite{Partadiredja}
sind Beispiele für Werke, die bereits vorhandene Richtlinien und ähnliche Werke zu ethischen Festlegungen bezüglich KI 
aufgreifen und darlegen, wobei \cite{Partadiredja} auch Unterschiede zwischen von KI und von Menschen generierte 
Mediacontent einschließlich Bildern herausarbeitet und dabei insbesondere auf ethische Implikationen eingeht. 
Es gibt solche weisenden, jedoch nicht gesetzlich bindenden Vorschriften also, jedoch demonstrieren Werke wie 
\cite{Ayling}, dass diese bei der Entwicklung und dem Einsatz von KI noch nicht ausreichend praktische Anwendung 
finden. \cite{Ayling} zeigt hierzu explizit Defizite aktueller Werkzeuge für Audits und Risikobewertungen bezüglich 
ethischer Rahmenwerke und Grundsätze in der KI auf, die in Zukunft berücksichtigt werden sollten. 
Prinzipiell gibt es bereits Arbeiten, deren Ergebnisse die Umsetzung verschiedener Richtlinien fördern können, so legt 
\cite{EUCommision} Anforderungen an KI in der EU fest. \cite{Jobin} und \cite{Hagendorff} definieren Strategien 
für die Implementierung von Richtlinien für ethische Prinzipien für KI. \cite{Stahl} diskutiert die Ethik von KI und Robotik 
und liefert Einblicke, wie ethische Prinzipien im Allgemeinen in der Praxis angewendet werden können. \cite{Srinivasan} bietet 
eine Zusammenstellung von Bias in verschiedenen Phasen des KI-Prozesses und gibt Empfehlungen für bewährte Verfahren und 
Richtlinien zur Identifizierung und Minderung von Verzerrungen in KI-Systemen für Entwickler von maschinellem Lernen. \cite{Jameel} 
geht auf Grundlage einer Vorstellung verschiedener KI-Modelle auf bestimmte ethische Probleme ein und zeigt Wege auf, 
Daten von ethisch hoher Qualität zu erhalten.
Zudem geben einige Unternehmen Überblick über ihre Ansätze zur Verminderung von Verzerrungen bei der Benutzung von KI, 
beispielsweise IBM über \cite{Hobson}. In diesem Werk wird auch anhand von Prozessen dargelegt, wie ethisch korrekt mit KI 
umgegangen werden kann.
Eine Variante, bestimmte Inhalte zu blockieren, besteht in der Filterung. Einen solche Filter für Bilder beschreibt \cite{Zheng}.
Die beschriebene Software kann Haut auf Bildern erkennen und darüber z. B. nicht kinder- und jugendfreie Inhalte und Symbole herausfiltern.
Zusammenfassend lässt sich also sagen, dass die Problematik bei der Umsetzung ethischer Prinzipien in der KI erkannt und
durch Richtlinien u. ä. Anleitungen gegeben werden, wie diese geschehen kann. Außerdem gibt es Untersuchungen 
zur Anforderungsanalyse und Vorschläge zur Implementierung für diese Umsetzung, allerdings keinen zusammenfassenden 
Überblick, was denn tatsächlich praktisch getan wird, um Missbrauch bildergenerierender Kis zur Erzeugung ethisch 
fragwürdiger Inhalte zu verhindern.
Nennung und Einschätzung der verwendeten KI-Werkzeuge

Wie bereits in Kapitel 2 erwähnt, wurden für die Recherche verschiedene KI-Werkzeuge verwendet, deren Nutzung im Folgenden 
eingeschätzt und die Nützlichkeit dessen diskutiert werden soll. 
ChatGPT3 ist ein Sprachmodell und dient somit der Fomulierung sprachlich korrekter Texte, kann jedoch beispielsweise keine Paper 
zusammenfassen, sondern gibt nur anhand des Titels jeweils ähnlich klingende Zusammenfassungen, ohne sich dabei auf die tatsächlichen 
Inhalte des Papers berufen zu können. Somit eignet sich höchstens, um die betrachtete Disziplin erforschende Autoren zu finden 
und auf dieser Basis mit anderen Werkzeugen weiterzurecherchieren. Solche Unterstützung liefern allerdings auch wissenschaftlich
etabliertere Webseiten wie WikiCFP. Zudem könnte ChatGPT3 von Wissenschaftlern, die in der Sprache, in der sie eine wissenschaftliche 
Arbeit schreiben, nicht ausreichend versiert sind, als Unterstützung bei der Formulierung eines wohlklingenden Texts verwendet werden.
Neben dem reinen ChatGPT3 wurde Bing mit ChatGPT3 verwendet. Dies bietet gegenüber der oben beschriebenen Variante den Vorteil, 
dass Bing das Internet durchsucht und somit inhaltlich passendere Antworten liefern kann. Der Nachteil besteht darin, dass nur das 
Internet beipielsweise nach dem Papertitel und „summary“ durchsucht wird und eines der ersten Suchergebnisse in wohlformulierter 
Sprache widergegeben wird, weshalb auch hier keine wirklich brauchbaren Inhalte erzeugt werden.
Generell muss die Benutzung eines Werkzeugs geübt werden, um die Möglichkeiten dessen in guter Qualität auszuschöpfen. Demzufolge 
müsste länger und intensiver mit diesem Werkzeug gearbeitet werden, um eventuell nützlichere Ergebnisse zu erzielen als diese. Die 
oben genannte Einschätzung beruht nur auf einwöchige Nutzung ohne nennenswerte Vorkenntnisse und ist demnach nicht repräsentativ für 
das tatsächliche Potenzial der genutzten KI-Werkzeuge.
Dasselbe gilt für Elicit. Dieses Werkzeug ist durch die Größe seiner Datenbank beschränkt, innerhalb dieser jedoch hilfreich, um 
einen Überblick über Werke zu erlangen, deren Autoren nicht auf den hiesigen Konferenzen vertreten und folglich mithilfe der 
klassischen Recherche über Konferenzen und deren Paper und Teilnehmer nicht auffindbar wären. Mithilfe geeigneter Fragen kann 
auch Zeit bei der Recherche gespart werden, indem die Zusammenfassung der ersten vier Paper genutzt wird. Diese sollte jedoch 
nicht nur, wenn die eigentlich intendierte Frage nicht direkt beantwortet wird, überprüft werden, da sie sehr kurz und nicht 
immer vollständig zutreffend ist.

Die betrachteten Arbeiten lassen sich somit wie folgt inhaltlich gruppieren: Es gibt wissenschaftliche Arbeiten zur Analyse und Filterung
bestimmter Inputs/Outputs wie Haut und Symbole aus Bildern \cite{Zheng} oder unmoralische Wörter oder Phrasen \cite{Shah}. Des weiteren 
werden Richtlinien zum ethischen Einsatz von KI zusammenstellt oder definiert und diskutiert \cite{Ayling} \cite{Srinivasan} 
\cite{Jameel} \cite{Hagendorff} \cite{Jobin} \cite{Unity} \cite{EUCommision}[World Economic Forum 2018] \cite{Mueller}.
Neben diesen Richtlinien werden Frameworks für KI entwickelt [Huang et al. 2022] \cite{Mueller}. Es werden Probleme von Kis ermittelt \cite{Ayling}, 
beispielsweise Bias \cite{Salminen} \cite{Jameel}, Kopien \cite{Somepalli}, wobei u. a. Bias auch zu ethisch 
fragwürdigen Resultaten führen kann [Zuber 2022]. Lösungsansätze für diese Probleme werden allgemein für KI-spezifische Probleme ohne 
Konzentration auf ein bestimmtes \cite{Ayling} \cite{Avelar}, aber auch konkret für bestimmte Problemkategorien wie Bias \cite{Srinivasan}
\cite{Jameel}untersucht. Technikfolgen und Regulierungsfragen für verschiedene von KI beeinflusste Kontexte wie Politik und Wirtschaft
werden analysiert \cite{Pawelec}.

\subsection{Methoden der Quellen}
Die meisten Quellen haben entweder empirisch gearbeitet oder eine Literatur\"ubersicht erarbeitet. Einige haben jedoch keine Methodiken genannt wie \cite{Unity}  