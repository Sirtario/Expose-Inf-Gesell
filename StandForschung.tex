\chapter{Stand der Forschung}

Nachfolgend wird der Stand der Forschung zum Thema Möglichkeiten zur Verhinderung 
des Missbrauchs bildergenerierender Kis zur Erzeugung ethisch fragwürdiger Inhalte mit Fokus darauf beleuchtet, wie in der Praxis verhindert wird, dass \GLSabrev{igai} zur Erzeugung ethisch fragwürdiger Inhalte missbraucht werden. Anschließend wird die Nützlichkeit der für die Recherche verwendeten \GLSabrev{ai}-Anwendungen kurz diskutiert.

Werke wie \cite{Salminen} zeigen, dass \GLSabrev{ai} nicht ethisch ist, sondern Ergebnisse auf Basis ihres Datensatzes liefert,
und Bias schnell zu diskriminierenden Verteilungen von Ergebnismengen führen können. \cite{Jobin} und \cite{Partadiredja}
sind Beispiele für Werke, die bereits vorhandene Richtlinien und ähnliche Werke zu ethischen Festlegungen bezüglich \GLSabrev{ai} 
aufgreifen und darlegen, wobei \cite{Partadiredja} auch Unterschiede zwischen von \GLSabrev{ai} und von Menschen generierte 
Mediacontent einschließlich Bildern herausarbeitet und dabei insbesondere auf ethische Implikationen eingeht. 
Es gibt solche weisenden, jedoch nicht gesetzlich bindenden Vorschriften also, jedoch demonstrieren Werke wie 
\cite{Ayling}, dass diese bei der Entwicklung und dem Einsatz von \GLSabrev{ai} noch nicht ausreichend praktische Anwendung 
finden. \cite{Ayling} zeigt hierzu explizit Defizite aktueller Werkzeuge für Audits und Risikobewertungen bezüglich 
ethischer Rahmenwerke und Grundsätze in der \GLSabrev{ai} auf, die in Zukunft berücksichtigt werden sollten. 
Prinzipiell gibt es bereits Arbeiten, deren Ergebnisse die Umsetzung verschiedener Richtlinien fördern können, so legt 
\cite{EUCommision} Anforderungen an \GLSabrev{ai} in der EU fest. \cite{Jobin} und \cite{Hagendorff} definieren Strategien 
für die Implementierung von Richtlinien für ethische Prinzipien für \GLSabrev{ai}. \cite{Stahl} diskutiert die Ethik von \GLSabrev{ai} und Robotik 
und liefert Einblicke, wie ethische Prinzipien im Allgemeinen in der Praxis angewendet werden können. \cite{Srinivasan} bietet 
eine Zusammenstellung von Bias in verschiedenen Phasen des \GLSabrev{ai}-Prozesses und gibt Empfehlungen für bewährte Verfahren und 
Richtlinien zur Identifizierung und Minderung von Verzerrungen in \GLSabrev{ai}-Systemen für Entwickler von maschinellem Lernen. \cite{Jameel} 
geht auf Grundlage einer Vorstellung verschiedener \GLSabrev{ai}-Modelle auf bestimmte ethische Probleme ein und zeigt Wege auf, 
Daten von ethisch hoher Qualität zu erhalten.
Zudem geben einige Unternehmen Überblick über ihre Ansätze zur Verminderung von Verzerrungen bei der Benutzung von \GLSabrev{ai}, 
beispielsweise IBM über \cite{Hobson}. In diesem Werk wird auch anhand von Prozessen dargelegt, wie ethisch korrekt mit \GLSabrev{ai} 
umgegangen werden kann.
Eine Variante, bestimmte Inhalte zu blockieren, besteht in der Filterung. Einen solche Filter für Bilder beschreibt \cite{Zheng}.
Die beschriebene Software kann Haut auf Bildern erkennen und darüber z. B. nicht kinder- und jugendfreie Inhalte und Symbole herausfiltern.
Zusammenfassend lässt sich also sagen, dass die Problematik bei der Umsetzung ethischer Prinzipien in der \GLSabrev{ai} erkannt und
durch Richtlinien u. ä. Anleitungen gegeben werden, wie diese geschehen kann. Außerdem gibt es Untersuchungen 
zur Anforderungsanalyse und Vorschläge zur Implementierung für diese Umsetzung, allerdings keinen zusammenfassenden 
Überblick, was denn tatsächlich praktisch getan wird, um Missbrauch bildergenerierender Kis zur Erzeugung ethisch 
fragwürdiger Inhalte zu verhindern.
Nennung und Einschätzung der verwendeten \GLSabrev{ai}-Werkzeuge

Die betrachteten Arbeiten lassen sich somit wie folgt inhaltlich gruppieren: Es gibt wissenschaftliche Arbeiten zur Analyse und Filterung
bestimmter Inputs/Outputs wie Haut und Symbole aus Bildern \cite{Zheng} oder unmoralische Wörter oder Phrasen \cite{Shah}. Des weiteren 
werden Richtlinien zum ethischen Einsatz von \GLSabrev{ai} zusammenstellt oder definiert und diskutiert \cite{Ayling} \cite{Srinivasan} 
\cite{Jameel} \cite{Hagendorff} \cite{Jobin} \cite{Unity} \cite{EUCommision}[World Economic Forum 2018] \cite{Mueller}.
Neben diesen Richtlinien werden Frameworks für \GLSabrev{ai} entwickelt [Huang et al. 2022] \cite{Mueller}. Es werden Probleme von Kis ermittelt \cite{Ayling}, 
beispielsweise Bias \cite{Salminen} \cite{Jameel}, Kopien \cite{Somepalli}, wobei u. a. Bias auch zu ethisch 
fragwürdigen Resultaten führen kann [Zuber 2022]. Lösungsansätze für diese Probleme werden allgemein für \GLSabrev{ai}-spezifische Probleme ohne 
Konzentration auf ein bestimmtes \cite{Ayling} \cite{Avelar}, aber auch konkret für bestimmte Problemkategorien wie Bias \cite{Srinivasan}
\cite{Jameel}untersucht. Technikfolgen und Regulierungsfragen für verschiedene von \GLSabrev{ai} beeinflusste Kontexte wie Politik und Wirtschaft
werden analysiert \cite{Pawelec}.
